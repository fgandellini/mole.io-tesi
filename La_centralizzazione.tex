L'azienda che ha ospitato lo \textit{stage} di Tesi, Codice Plastico, � una realt� che opera nel mercato IT sviluppando applicazioni su misura di tipo \textit{mobile}, \textit{web} e \textit{desktop}.
Il contesto dal quale � nato lo sviluppo di questo progetto � stato la necessit� di trovare una soluzione per centralizzare i log in un unico sistema, facilmente accessibile e con un'interfaccia utente \textit{user-friendly}. 

Ogni applicazione realizzata e posata dall'azienda, infatti, presenta criticit� differenti rispetto alla gestione degli errori:
\begin{itemize}
\item Nelle applicazioni web, di norma, i log risiedono su server di propriet� dei  clienti, spesso dislocati in aree geografiche differenti.
\item Nel caso delle applicazioni mobile, i log risiedono sui device stessi, cos� come nel caso delle applicazioni desktop.
\end{itemize}

Tra i numerosi vantaggi della centralizzazione il principale � la riduzione del carico di lavoro dell'analista, il quale pu� avere accesso ai log senza l'onere di doverli attivamente cercare sui sistemi dei clienti.

Inoltre l'aggregazione dei dati permette all'analista di aver accesso ad informazioni gi� pre-elaborate, come ad esempio il \textit{clustering} di errori simili avvenuti in istanti temporali differenti con l'evidente semplificazione del processo di riconoscimento delle correlazioni causa-effetto.

Il processo di centralizzazione richiede l'inversione del paradigma di segnalazione degli errori. Si passa da una modalit� nella quale � l'analista a dover cercare attivamente i file sui sistemi, ad una in cui sono i sistemi stessi ad inviare i propri log. A questa logica � facilmente integrabile un sistema di notifiche in tempo reale, con lo scopo di rendere repentina la segnalazione dell'errore e il conseguente intervento di ripristino.

In commercio esistono svariati sistemi e \textit{tool} che consentono la gestione delle problematiche legate ai log, come ad esempio l'archiviazione, l'analisi, il \textit{parsing}, il monitoraggio e le segnalazioni. Anche in questo caso la centralizzazione degli strumenti di controllo, evita  l'installazione e conseguente manutenzione di numerosi tool su ciascun sistema in produzione. 

L'effetto evidente della centralizzazione, della semplificazione dell'accesso ai dati e della loro analisi � la riduzione di costi di manutenzione del software, oltre all'incremento della qualit� del prodotto offerto.




