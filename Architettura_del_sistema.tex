Come abbiamo anticipato nell'introduzione di questo capitolo, Mole.io � organizzato in diverse componenti, ognuna delle quali possiede un compito specifico e interagisce con altre.

Nell'architettura di Mole.io si possono innanzitutto individuare due macro-componenti principali. Nelle sezioni seguenti ne illustreremo la struttura.

%TODO cerca in giro i riferimenti a Architettura_del_sistema

%TODO parlare di REST, spiegare uno per uno gli endpoint della nostra rest api

\subsubsection{Insertion}

Questo \textit{layer} dell'applicazione si occupa dell'inserimento dei dati nel sistema e si avvale di diversi moduli per fare in modo che gli whisper vengano salvati, aggregati e organizzati nel database.

Nello schema \ref{} � riportata l'architettura del layer di insertion, essa � composta da diversi moduli:

mole-contacts

la porzione di sistema che si occupa dell'inserimento dei dati provenienti dall'esterno, � composta a sua volta dai \textit{mole-contacts}, da \textit{mole} e dai \textit{denormalizers}.



\begin{description}
\item [insertion] � la porzione di sistema che si occupa dell'inserimento dei dati provenienti dall'esterno, � composta a sua volta dai \textit{mole-contacts}, da \textit{mole} e dai \textit{denormalizers}.
\item [presentation] si occupa dell'estrazione dei dati dal sistema e della loro presentazione all'utente. 
\end{description}






% dire che esistono 2 server
% dire che la ui � separata e statica e pu� essere servita da un terzo server
% configurazione rabbit utilizzata per scalare (broadcast+loadbalancing)

% middleware