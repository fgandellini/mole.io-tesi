In questa tesi si descriver� Mole.io: un sistema centralizzato per la raccolta e l'aggregazione di messaggi provenienti da applicazioni remote.

Durante il loro ciclo di lavoro o \textit{processing}, le applicazioni software eseguono operazioni significative o entrano in situazioni di errore. In questi casi � importante che le persone che hanno in carico la gestione di questi sistemi, siano informate dell'accaduto in modo da operare scelte opportune o applicare le dovute correzioni (\textit{bugfix}).

Gli sviluppatori sono soliti utilizzare messaggi di tracciamento (\textit{log}) per stampare a video o salvare in \textit{file} stati significativi delle applicazioni. I messaggi pi� frequenti riportati nei log sono quelli relativi a situazioni di errore (\textit{Exception} e \textit{Stack Trace}).

L'approccio comune alla creazione e gestione dei log presenta la criticit� specifica della \textit{localit�}, poich� tipicamente questi file vengono salvati nella stessa macchina sulla quale sta operando l'applicazione.

All'aumentare del numero di applicazioni da gestire e del numero di macchine in produzione, capita spesso che i server siano in luoghi geograficamente distanti tra loro. Questa situazione rende evidente la difficolt� di ottenere un \textit{feedback} veloce dello stato di ogni software e delle eventuali situazioni di errore in cui le applicazioni si trovano.

Mole.io cerca di risolvere il problema facendo in modo che i software che lo utilizzano, siano in grado di inviare le informazioni che ritengono significative ad un server centrale, il quale le raccoglie, le cataloga e le aggrega per essere facilmente supervisionate da parte degli sviluppatori.

L'esigenza di una applicazione per la centralizzazione dei log nasce da CodicePlastico \cite{website:CodicePlastico}, una azienda con sede a Brescia, che si occupa di realizzare applicazioni su misura per i propri clienti. 

La gestione di un gran numero installazioni dislocate sul territorio e di molte realt� aziendali con esigenze differenti ha reso, per CodicePlastico, particolarmente complesso il tracciamento dello stato di ogni software in produzione. Questa situazione ha spinto l'azienda a decidere di dotarsi di un sistema centralizzato in grado di collezionare i log prodotti dai diversi applicativi, organizzarli e catalogarli in modo automatico. 

Il nuovo approccio permette agli sviluppatori di identificare, in breve tempo, il manifestarsi di un malfunzionamento in qualunque applicazione installata presso uno dei proprio clienti e reagire rapidamente proponendo una azione risolutiva.

CodicePlastico opera nel settore IT avvalendosi di strumenti software di vario genere, sia proprietari, sia \textit{open source}. Per il \textit{deploy} in produzione di Mole.io, si � scelto di utilizzare \textit{Microsoft Azure}, un sistema \textit{PaaS} distribuito con il quale � possibile creare architetture facilmente scalabili per supportare la variabilit� del carico di lavoro richiesto al sistema. 

mettila gi� cos�: in un fase di test il sistema sar� testato su un numero X di installazione, con in progetto di aumentare il campione successivamente... non siamo umili a priori ;-)

Durante una prima fase di \textit{test}, Mole.io sar� utilizzato per gestire tre o quattro installazioni, che fungeranno da \textit{pilota} per il progetto. La pianificazione dell'azienda, per l'immediato futuro, prevede di aumentare velocemente il numero di clienti attivi nel sistema. Una architettura scalabile, di conseguenza, permetter� di affrontare le esigenze di reattivit� e stabilit� del sistema al variare del carico di lavoro.

Nel primo capitolo saranno trattati approfonditamente la tematica dei log, i contesti nei quali essi vengono utilizzati e le problematiche legate alla gestione di questo tipo di soluzione di tracciamento. Sar� analizzato anche come utilizzare i log per ottenere informazioni di supporto alla \textit{business intelligence}. 

Il secondo capitolo riporter� un elenco dei principali \textit{software} per la gestione centralizzata dei log presenti sul mercato e delle soluzioni \textit{Open Source} che sono state prese a modello per la realizzazione di Mole.io. Verr� descritta ogni applicazione e sar� mostrato come Mole.io possa essere una soluzione innovativa sotto svariati punti di vista.

I due capitoli seguenti permetteranno di approfondire i dettagli tecnici delle metodologie di sviluppo applicate durante il \textit{design} del software e alcune tra le principali tecnologie utilizzate per la realizzazione del sistema.

Il quinto capitolo descriver� la struttura di Mole.io e le varie componenti software che rendono l'applicazione scalabile e garantiscono l'alta accessibilit� della soluzione.

Nel sesto capitolo verr� mostrato in modo oggettivo, con \textit{benchmark} e \textit{stress test} il comportamento di Mole.io all'aumentare del carico di lavoro e sar� dimostrato come le soluzioni di design applicate garantiscano buone \textit{performance}, anche in condizioni critiche di traffico.

Infine verranno discussi i risultati ottenuti e saranno proposte alcune interessanti funzionalit� che trasformeranno Mole.io dall'attuale \textit{proof of concept} ad un vero e proprio servizio.
