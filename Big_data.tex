sss

% Big Data
%Big data[1][2] is the term for a collection of data sets so large and complex that it becomes difficult to process using on-hand database management tools or traditional data processing applications. The challenges include capture, curation, storage,[3] search, sharing, transfer, analysis[4] and visualization. The trend to larger data sets is due to the additional information derivable from analysis of a single large set of related data, as compared to separate smaller sets with the same total amount of data, allowing correlations to be found to "spot business trends, determine quality of research, prevent diseases, link legal citations, combat crime, and determine real-time roadway traffic conditions."[5][6][7]

% grafico http://www.martinhilbert.net/WorldInfoCapacity.html

%As of 2012, limits on the size of data sets that are feasible to process in a reasonable amount of time were on the order of exabytes of data.[8] Scientists regularly encounter limitations due to large data sets in many areas, including meteorology, genomics,[9] connectomics, complex physics simulations,[10] and biological and environmental research.[11] The limitations also affect Internet search, finance and business informatics. Data sets grow in size in part because they are increasingly being gathered by ubiquitous information-sensing mobile devices, aerial sensory technologies (remote sensing), software logs, cameras, microphones, radio-frequency identification readers, and wireless sensor networks.[12][13][14] The world's technological per-capita capacity to store information has roughly doubled every 40 months since the 1980s;[15] as of 2012, every day 2.5 exabytes (2.5×1018) of data were created.[16] The challenge for large enterprises is determining who should own big data initiatives that straddle the entire organization.[17]
%Big data is difficult to work with using most relational database management systems and desktop statistics and visualization packages, requiring instead "massively parallel software running on tens, hundreds, or even thousands of servers".[18] What is considered "big data" varies depending on the capabilities of the organization managing the set, and on the capabilities of the applications that are traditionally used to process and analyze the data set in its domain. "For some organizations, facing hundreds of gigabytes of data for the first time may trigger a need to reconsider data management options. For others, it may take tens or hundreds of terabytes before data size becomes a significant consideration."[19]

% ---------------------------------------

%Big Data usually includes data sets with sizes beyond the ability of commonly used software tools to capture, curate, manage, and process the data within a tolerable elapsed time.[20] Big data sizes are a constantly moving target, as of 2012 ranging from a few dozen terabytes to many petabytes of data in a single data set.
%In a 2001 research report[21] and related lectures, META Group (now Gartner) analyst Doug Laney defined data growth challenges and opportunities as being three-dimensional, i.e. increasing volume (amount of data), velocity (speed of data in and out), and variety (range of data types and sources). Gartner, and now much of the industry, continue to use this "3Vs" model for describing big data.[22] In 2012, Gartner updated its definition as follows: "Big data is high volume, high velocity, and/or high variety information assets that require new forms of processing to enable enhanced decision making, insight discovery and process optimization."[23] Additionally, a new V "Veracity" is added by some organizations to describe it.[24]
%If Gartner?s definition (the 3Vs) is still widely used, the growing maturity of the concept fosters a more sound difference between big data and Business Intelligence, regarding data and their use:
%Business Intelligence uses descriptive statistics with data with high information density to measure things, detect trends etc.;
%Big data uses inductive statistics and concepts from nonlinear system identification [25] to infer laws (regressions, nonlinear relationships, and causal effects) from large data sets [26] to reveal relationships, dependencies, and to perform predictions of outcomes and behaviors.[25][27]