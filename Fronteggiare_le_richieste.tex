Come � stato anticipato, MongoDB � un database documentale che garantisce alte \textit{performance}, alta accessibilit� e permette facilmente di scalare la sua struttura per fronteggiare le richieste. Di seguito sar� descritto brevemente come MongoDB realizza ognuna di queste funzionalit�:

\begin{description}
\item[Database documentale] I documenti contenuti in MongoDB (oggetti) mappano molto bene gli oggetti e le strutture dati fornite dai principali linguaggi di programmazione, rendendo quasi inutile la necessit� di un software di traduzione tra strutture dati utilizzate nel software e la loro rappresentazione nel database. Tipicamente i software di \textit{Object-Relational Mapping} (ORM) sono necessari in presenza di database relazionali. Il vantaggio di avere gli \textit{embedded document}, inoltre, permette di ridurre il numero di operazioni di \textit{join} sui dati, rendendo superflua una delle caratteristiche fondamentali dei RDBMS. Con MongoDB diventa quindi molto semplice far \textit{evolvere} le proprie strutture dati, rendendo lo sviluppo dell'applicazione molto pi� flessibile.

\item[Alte \textit{performance}] La possibilit� di inserire documenti all'interno di altri documenti, garantisce scritture veloci e la definizione degli indici pu� includere chiavi presenti negli \textit{embedded documents}, in questo modo � possibile ottenere tempi di risposta del sistema molto contenuti.

\item[Alta accessibilit�] Questa propriet�, chiamata tecnicamente, \textit{High Availability} (HA), � realizzata con server replicati e organizzati in \textit{cluster}, detti \textit{replica-set}, nei quali � possibile identificare un \textit{master} e altri \textit{slave}. Il master riceve le richieste e le smista sugli slave nel cluster. In caso di problemi al server master, MongoDB esegue una elezione automatica del nuovo master tra gli slave rimanenti.

\item[Facile scalabilit�] Lo \textit{sharding} � la possibilit� di suddividere una collection su pi� server in modo automatico. Questa caratteristica rende MongoDB altamente indicato per essere installato su piattaforme di tipo PaaS, all'aumentare dei dati presenti nel database, infatti, � sufficiente aumentare il numero di server a disposizione per accogliere pi� informazioni. Dal punto di vista dell'applicazione che utilizza questi dati, questa operazione � trasparente. In questo modo il numero di server necessari aumenta linearmente all'aumentare della quantit� di dati salvata. \`{E} possibile aggiungere server in modo dinamico, senza arrestare il sistema, questa funzionalit� � particolarmente importante quando MongoDB � utilizzato per contenere dati di applicazioni web che non ammettono momenti di \textit{downtime}. 
%parlare di eventual consistency? \cite{Vogels:2008:EC:1466443.1466448}
%Eventually-consistent reads can be distributed over replicated servers.
\end{description}

Oltre alla flessibilit� offerta dal modello documentale, MongoDB include le funzionalit� comuni degli RDBMS, quali indici, aggregazioni, \textit{query}, ordinamenti, aggiornamenti di dati aggregati e \textit{upsert}, cio� aggiornamento di un dato se gi� esistente o creazione di un nuovo dato. 

Gli sviluppatori di MongoDB hanno cercato di realizzare un database che fosse semplice da installare e manutenere, infatti la filosofia alla base di MongoDB � "fare la cosa giusta". Questo significa che il sistema cerca di adattarsi nel miglior modo possibile alla configurazione dell'\textit{hardware} che lo ospita e fornisce un insieme limitato di parametri di configurazione, mantenendo cos� una interfaccia \textit{user-friendly} verso gli amministratori e permettendo agli sviluppatori di concentrarsi sulle logiche applicative invece di occuparsi della configurazione del database.

Sebbene MongoDB supporti configurazioni \textit{standalone} (o \textit{single-instance}), la configurazione comune di questo database in produzione � quella distribuita. Combinando le funzionalit� di \textit{replica-set} e \textit{sharding} � possibile ottenere alti livelli di ridondanza per grandi basi di dati in modo completamente trasparente per l'applicazione.