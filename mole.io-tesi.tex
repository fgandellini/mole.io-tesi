% Autore: Federico Gandellini
% Titolo: mole.io

% Documento
\documentclass[a4paper,11pt,italian]{report}
%\documentclass[a4paper,11pt,italian,draft]{report} % Per non caricare le immagini

% Packages
\usepackage[italian]{babel}
\usepackage[latin1]{inputenc}
\usepackage{graphicx}
\usepackage{color}
\usepackage{colortbl}
\usepackage{amsmath}
\usepackage{amsfonts}
\usepackage{rotating}
% Variabili

% Intestazione
%\usepackage{fancyhdr}
% Line spacing -----------------------------------------------------------
\newlength{\defbaselineskip}
\setlength{\defbaselineskip}{\baselineskip}
\newcommand{\setlinespacing}[1]%
           {\setlength{\baselineskip}{#1 \defbaselineskip}}
\newcommand{\doublespacing}{\setlength{\baselineskip}%
                           {2.0 \defbaselineskip}}
\newcommand{\singlespacing}{\setlength{\baselineskip}{\defbaselineskip}}

%-----------------------------FANCYHDR--------------------------------------%
\pagestyle{fancy}                       % Sets fancy header and footer
\fancyfoot{}                            % Delete current footer settings
\renewcommand{\chaptermark}[1]{         % Lower Case Chapter marker style
  \markboth{\chaptername\ \thechapter.\ #1}{}} %
\renewcommand{\sectionmark}[1]{         % Lower case Section marker style
  \markright{\thesection.\ #1}}         %
\fancyhead[LE,RO]{\bfseries\thepage}    % Page number (boldface) in left on even
                                        % pages and right on odd pages
\fancyhead[RE]{\bfseries\leftmark}      % Chapter in the right on even pages
\fancyhead[LO]{\bfseries\rightmark}     % Section in the left on odd pages
\renewcommand{\headrulewidth}{0.05pt}    % Width of head rule
%\pagestyle{fancy}\addtolenght{\headwidth}{20pt}
%\rnewcommand{\chaptermark}[1]{\markboth{\thechapter.\#1}{}}
%\rnewcommand{\sectionmark}[1]{\markright{\thesection\#1}{}}
%\cfoot{}
%\rhead[\fancyplain{}{\bfseries\leftmark}]{\fancyplain{}{\bfseries\thepage}}
%\lhead[\fancyplain{}{\bfseries\thepage}]{\fancyplain{}{\bfseries\rightmark}}


% Formattazione
\linespread{1.6} % Interlinea 2

\begin{document}

%\input{Intestazione}
\clearpage
%\begin{titlepage}

\setlength{\topmargin}{1pt} \setlength{\headheight}{1pt}

\huge{\textbf{Ringraziamenti}}

\vspace{3.0cm} \normalsize \noindent Un grazie a ...

% parenti e amici
% codice plastico
% chi fa sw open

\end{titlepage}
\clearpage
\tableofcontents
\clearpage

\chapter*{Introduzione\markboth{}{Introduzione}}
\addcontentsline{toc}{chapter}{Introduzione}
%In questa tesi si descriver� Mole.io: un sistema centralizzato per la raccolta e l'aggregazione di messaggi provenienti da applicazioni remote.

Durante il loro ciclo di lavoro o \textit{processing}, le applicazioni software eseguono operazioni significative o entrano in situazioni di errore. In questi casi � importante che le persone che hanno in carico la gestione di questi sistemi, siano informate dell'accaduto in modo da operare scelte opportune o applicare le dovute correzioni (\textit{bugfix}).

Gli sviluppatori sono soliti utilizzare messaggi di tracciamento (\textit{log}) per stampare a video o salvare in \textit{file} stati significativi delle applicazioni. I messaggi pi� frequenti riportati nei log sono quelli relativi a situazioni di errore (\textit{Exception} e \textit{Stack Trace}).

L'approccio comune alla creazione e gestione dei log presenta la criticit� specifica della \textit{localit�}, poich� tipicamente questi file vengono salvati nella stessa macchina sulla quale sta operando l'applicazione.

All'aumentare del numero di applicazioni da gestire e del numero di macchine in produzione, capita spesso che i server siano in luoghi geograficamente distanti tra loro. Questa situazione rende evidente la difficolt� di ottenere un \textit{feedback} veloce dello stato di ogni software e delle eventuali situazioni di errore in cui le applicazioni si trovano.

Mole.io cerca di risolvere il problema facendo in modo che i software che lo utilizzano, siano in grado di inviare le informazioni che ritengono significative ad un server centrale, il quale le raccoglie, le cataloga e le aggrega per essere facilmente supervisionate da parte degli sviluppatori.

L'esigenza di una applicazione per la centralizzazione dei log nasce da CodicePlastico \cite{website:CodicePlastico}, una azienda con sede a Brescia, che si occupa di realizzare applicazioni su misura per i propri clienti. 

La gestione di un gran numero installazioni dislocate sul territorio e di molte realt� aziendali con esigenze differenti ha reso, per CodicePlastico, particolarmente complesso il tracciamento dello stato di ogni software in produzione. Questa situazione ha spinto l'azienda a decidere di dotarsi di un sistema centralizzato in grado di collezionare i log prodotti dai diversi applicativi, organizzarli e catalogarli in modo automatico. 

Il nuovo approccio permette agli sviluppatori di identificare, in breve tempo, il manifestarsi di un malfunzionamento in qualunque applicazione installata presso uno dei proprio clienti e reagire rapidamente proponendo una azione risolutiva.

CodicePlastico opera nel settore IT avvalendosi di strumenti software di vario genere, sia proprietari, sia \textit{open source}. Per il \textit{deploy} in produzione di Mole.io, si � scelto di utilizzare \textit{Microsoft Azure}, un sistema \textit{PaaS} distribuito con il quale � possibile creare architetture facilmente scalabili per supportare la variabilit� del carico di lavoro richiesto al sistema. 

mettila gi� cos�: in un fase di test il sistema sar� testato su un numero X di installazione, con in progetto di aumentare il campione successivamente... non siamo umili a priori ;-)

Durante una prima fase di \textit{test}, Mole.io sar� utilizzato per gestire tre o quattro installazioni, che fungeranno da \textit{pilota} per il progetto. La pianificazione dell'azienda, per l'immediato futuro, prevede di aumentare velocemente il numero di clienti attivi nel sistema. Una architettura scalabile, di conseguenza, permetter� di affrontare le esigenze di reattivit� e stabilit� del sistema al variare del carico di lavoro.

Nel primo capitolo saranno trattati approfonditamente la tematica dei log, i contesti nei quali essi vengono utilizzati e le problematiche legate alla gestione di questo tipo di soluzione di tracciamento. Sar� analizzato anche come utilizzare i log per ottenere informazioni di supporto alla \textit{business intelligence}. 

Il secondo capitolo riporter� un elenco dei principali \textit{software} per la gestione centralizzata dei log presenti sul mercato e delle soluzioni \textit{Open Source} che sono state prese a modello per la realizzazione di Mole.io. Verr� descritta ogni applicazione e sar� mostrato come Mole.io possa essere una soluzione innovativa sotto svariati punti di vista.

I due capitoli seguenti permetteranno di approfondire i dettagli tecnici delle metodologie di sviluppo applicate durante il \textit{design} del software e alcune tra le principali tecnologie utilizzate per la realizzazione del sistema.

Il quinto capitolo descriver� la struttura di Mole.io e le varie componenti software che rendono l'applicazione scalabile e garantiscono l'alta accessibilit� della soluzione.

Nel sesto capitolo verr� mostrato in modo oggettivo, con \textit{benchmark} e \textit{stress test} il comportamento di Mole.io all'aumentare del carico di lavoro e sar� dimostrato come le soluzioni di design applicate garantiscano buone \textit{performance}, anche in condizioni critiche di traffico.

Infine verranno discussi i risultati ottenuti e saranno proposte alcune interessanti funzionalit� che trasformeranno Mole.io dall'attuale \textit{proof of concept} ad un vero e proprio servizio.

\clearpage

\chapter{Log: contesti e problematiche}\label{Log_contesti_e_problematiche}
	\section{La centralizzazione}\label{Centralizzare_i_log}
	\clearpage
	\section{Trattare gli errori applicativi}\label{Trattare_gli_errori_applicativi}
	\clearpage
	\section{Business intelligence}\label{Businness_Intelligence}
    \clearpage

\chapter{Log: software e applicazioni}\label{Log_software_e_applicazioni}
	\section{Prodotti e soluzioni sul mercato}\label{Prodotti_e_soluzioni_sul_mercato}
	\clearpage
	\section{Una nuova applicazione: Mole.io}\label{Una_nuova_applicazione_Mole.io}
	\clearpage

\chapter{Metodologie di sviluppo}\label{Metodologie_di_sviluppo}
	\section{User stories}\label{User_stories}
	\clearpage
	\section{Test e behavior driven development}\label{Test_e_behavior_driven_development}
	\clearpage

\chapter{Tecnologie utilizzate}\label{Tecnologie_utilizzate}
	\section{Node.js}\label{Node.js}
		% motivo della scelta
		\subsection{La storia}\label{La_storia}
		\clearpage
		\subsection{NPM e moduli}\label{NPM_e_moduli}
		\clearpage
	\section{RabbitMQ}\label{RabbitMQ}
	% motivo della scelta
	\clearpage
	\section{MongoDB}\label{MongoDB}
	% motivo della scelta
		\subsection{Fronteggiare le richieste}\label{Fronteggiare_le_richieste}
		\clearpage
	\clearpage	
	\section{AngularJS e altre tecnologie di frontend}\label{AngularJS e altre tecnologie di frontend}
	% motivo della scelta
		\subsection{Gestione delle dipendenze}\label{Gestione_delle_dipendenze}
		\clearpage
	\section{Strumenti per il deploy}\label{Strumenti_per_il_deploy}
	\clearpage

\chapter{Mole.io}\label{Mole.io}
	\section{Architettura del sistema}\label{Architettura_del_sistema}
		\subsection{CQRS ed estensibilit�}\label{CQRS_ed_estensibilita}
		\clearpage
		\subsection{mole}\label{mole}
			\subsubsection{I denormalizzatori}\label{I_denormalizzatori}
			\clearpage
		\subsection{mole-suit}\label{mole-suit}
			\subsubsection{I plugin e gli widget}\label{I_plugin_e_gli_widget}
			\clearpage
	\section{Autenticazione degli utenti}\label{Autenticazione_degli_utenti}
	\clearpage
	\section{Scalabilit� e affidabilit�}\label{Scalabilita_e_affidabilita}
	\clearpage
	\section{Problematiche di sviluppo}\label{Problematiche_di_sviluppo}
	\clearpage

\chapter{Configurazioni e benchmark}\label{Configurazioni_e_benchmark}
\clearpage

\chapter*{Conclusioni e sviluppi futuri\markboth{}{Conclusioni e sviluppi futuri}}\label{Conclusioni_e_sviluppi_futuri}
\addcontentsline{toc}{chapter}{Conclusioni e sviluppi futuri}
%testere il sistema "in piccolo" � difficile, provare test sul sistema deployato

%TODO
% workflow con webhooks
% notifiche push (si potrebbe usare un denormalizer?)
% app mobile per riceverle
% https
% marcatura degli errori come "gestito" e conseguente archiviazione
% integrazione con sistemi di tracking (?)
% aggiungere pi� mole-contacts
% aggiungere un sistema di query
% mole-contacts -> caching degli errori in locale quando non c'� rete (buffering)

% migliorare la gestione dgli widget e plugin per l'aggiunta a caldo

% possibili nuovi denormalizzatori

% bench e vm sulla stessa macchina
% vm con cpu pin
% ramp up

\clearpage

\bibliography{Bibliografia}
\bibliographystyle{unsrt}
\addcontentsline{toc}{chapter}{Bibliografia}

\end{document}



%
%il problema
%- problema del logging:
%  - tante applicazioni
%  - diversi clienti
%  - report errori
%
%stato dell'arte: i software di logging centralizzato
%- aircoso
%- fluentd + elastic search + kibana
%perch\'{e} lo stato dell'arte non ci piace
%- non permettono di avere "modelli/template di messaggi" flessibili
%- contesti differenti
%
%il nostro logger 
%- perch\'{e} abbiamo scelto nodejs
%- cos'� nodejs (storia, peculiarit�, async, pesca da presentazione lucio)
%- come funziona il logger (struttura e schema delle comunicazioni di rete)
%- problematiche incontrate durante l'implementazione
%- si auto-adatta al tipo di messaggio in arrivo
%- denormalizzatori
%- plugin/widget
%- il clAud
%- architettura del sistema
%- estensibilit�
%- come scrivere plugin e widget
%
%problematiche "accademiche"
%- sicurezza (oauth)
%- scalabilit� (+ server con load balancer? cloud? rabbit, disaccoppiamento dei servizi)
%- relayability (+ server ridondati) 
%
%
%
%Scaletta:
%
%Metodologie:
% - CQRS
% - TDD
% - user stories
%
%Tecnologie utilizzate:
%- NodeJS
%  - npm (gestione dipendenze)
%  - passport (oauth)
%  - mongoskin
%  - mongoose
%  - underscore
%  - express
%  - faker (fixtures)
%  - grunt (automation)
%  - mocha (testing)
%  - should (testing)
%  - supertest (testing)
%  - require-all (plugin)
%  - cors (e problemi correlati)
%
%- rabbitmq
%
%- angularjs
%  - leflet (mappe)
%  - directive
%  - controller
%  - views
%  - services
%  - filters
%  - karma (testing)
%  - d3 (grafici)
%  - jquery
%  - bootstrap (ui)
%  - bower (gestione dipendenze)
%
%- mongodb
%  - sharding
%  - replicaset
%
%- dokku
%  
%- git(?)





