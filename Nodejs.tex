La \textit{homepage} del sito ufficiale \cite{Node.js} di Node.js fornisce una sintetica ma precisa descrizione di questa tecnologia, partiremo proprio da essa per illustrarne le peculiarit�.\\

La descrizione ufficiale recita:

\begin{center}
\textit{Node.js is a platform built on Chrome's JavaScript runtime for easily building fast, scalable network applications. Node.js uses an event-driven, non-blocking I/O model that makes it lightweight and efficient, perfect for data-intensive real-time applications that run across distributed devices.}
\end{center}

Scopriamo immediatamente che Node.js � una \textit{piattaforma}. L'utilizzo di questo termine mette l'accento su un aspetto fondamentale di questa tecnologia: fornire un ambiente nel quale le applicazioni sviluppate possano funzionare con il supporto di librerie di sistema fornite da Node.js stesso.

La piattaforma Node.js utilizza JavaScript come linguaggio di sviluppo. Per farlo si avvale del potente interprete \textit{V8} presente all'interno del browser \textit{Chrome} di \textit{Google}. L'utilizzo di un linguaggio altamente popolare e di una base solida come quella fornita dal popolare \textit{browser} permettono di costruire applicazioni in modo semplice e veloce.

L'ultimo importante concetto, che leggiamo dalla prima frase della descrizione, � che Node.js � principalmente orientato allo sviluppo di applicazioni che lavorano con la rete. Per sua natura, la piattaforma ci aiuta a fare in modo che esse siano scalabili.\\

La seconda parte della descrizione spiega sinteticamente alcune caratteristiche peculiari di Node.js e ne definisce meglio il contesto applicativo. 

L'intera piattaforma � centrata sul concetto di \textit{evento}. Si dice  evento un messaggio che viene scatenato in un determinato istante dell'elaborazione e che successivamente � catturato e gestito dalle componenti del sistema che sono preposte alla gestione di quell'evento specifico.

Il sistema ad eventi viene utilizzato da Node.js congiuntamente ad una gestione non bloccante delle operazioni di \textit{Input/Output}. Questo significa che una operazione potenzialmente lunga, come ad esempio la comunicazione con il \textit{filesystem} o con i dispositivi di \textit{rete} non bloccano il flusso di esecuzione del programma principale. Dopo aver richiesto ad altri attori del sistema il dato di cui necessita, il programma continua il suo normale flusso di esecuzione e verr� informato, utilizzando un evento, quando il dato richiesto sar� disponibile. Questa gestione non bloccante delle operazioni di \textit{I/O} � chiamata \textit{I/O} asincrono.

Questa modalit� di gestione delle operazioni non strettamente legate alla logica applicativa, permette a Node.js di essere molto efficiente se utilizzato per la realizzazione di applicazioni che elaborano grandi quantit� di dati ma devono rimanere \textit{reattive} nei confronti di nuove richieste di elaborazione.

\subsection{Le operazioni asincrone}

Abbiamo introdotto il concetto di I/O asincrono, di seguito illustreremo come Node.js riesca a gestirlo utilizzando una quantit� limitata di risorse di sistema.    

I componenti \textit{hardware} di un sistema elaborano dati con velocit� differenti. La rapidit� di ogni componente � fortemente legata al modo nel quale questo � stato realizzato. La tabella seguente riporta un elenco di componenti e le relative velocit� indicative riferite ad un singolo ciclo di \textit{CPU}. 

\begin{center}
\begin{tabular}{l|r}
\textbf{Componente} & \textbf{Numero di cicli} \\ 
\hline 
CPU & 1 \\ 
Cache di livello 1 & 3 \\ 
Cache di livello 2 & 14 \\ 
RAM & 250 \\ 
Hard Disk & 41.000.000 \\ 
Dispositivo di rete & 240.000.000 \\ 
\end{tabular} 
\end{center}

Guardando la tabella si nota immediatamente come i dispositivi di \textit{I/O} siano ordini di grandezza pi� lenti rispetto ai dispositivi di elaborazione delle informazioni. Se il flusso del programma dovesse aspettare in modo sincrono ogni singola operazione di lettura o scrittura su disco, ad esempio, esso perderebbe la possibilit� di eseguire circa quaranta milioni di operazioni di calcolo, con un conseguente degrado delle performances del software.

Una tecnica comune per affrontare il problema dell'\textit{I/O} � l'utilizzo di \textit{threads}. Un thread � spesso definito con il termine "sotto-processo leggero", in effetti esso condivide codice e risorse di sistema con altri threads appartenenti allo stesso processo padre.

Un thread permette al processo di parallelizzare l'esecuzione di una parte del suo flusso di lavoro, ma al tempo stesso sono necessarie risorse di sistema per creare il thread stesso e per distruggerlo. Node.js ovvia a questo problema utilizzando un \textit{thread pool}, cio� un insieme di threads che fungono da \textit{worker} per il processo. Quando � necessario eseguire un particolare task, esso viene sottoposto al thread pool, di conseguenza viene assegnato ad un worker, esso lo esegue e al termine del lavoro comunica l'esito dell'elaborazione al chiamante tramite una funzione detta \textit{callback}.

L'utilizzo di un thread pool comporta un incremento di prestazioni da parte delle applicazioni che lo sfruttano: i thread pool ottimizzano l'utilizzo della memoria e del processore, diminuendo l'overhead di gestione dei thread.

\subsection{Le callback}

Una delle principali difficolt� quando si utilizza Node.js per la prima volta � la gestione delle callback. Di seguito � riportato un frammento di codice Node.js che esegue il salvataggio di un dato su Database.

\begin{verbatim}
function save(user) {
  console.log('saving user into db');
  db.insert(user, function(err, user) {
    console.log('user successfully saved!');
  });
  console.log('all we need is just a little patience...');
}
\end{verbatim}

la funzione \verb|db.insert()| � asincrona, ed accetta due parametri: il dato da salvare e la callback da eseguire una volta salvato il dato. L'esecuzione di questo codice produce un output in console simile a quello mostrato qui sotto.

\begin{verbatim}
saving user into db
all we need is just a little patience...
user successfully saved!
\end{verbatim}

Se la funzione \verb|db.insert()| non fosse asincrona otterremmo un output come quello che segue.

\begin{verbatim}
saving user into db
user successfully saved!
all we need is just a little patience...
\end{verbatim}

Questo piccolo esempio, lascia intuire un potenziale problema nella gestione delle callback, se ad esempio dovessimo assegnare alcuni privilegi di \textit{default} a tutti i nuovi utenti inseriti nel sistema, potremmo scrivere il seguente codice.

\begin{verbatim}
function save(user) {
  console.log('saving user into db');
  db.insert(user, function(err, user) {
  	db.insert(grants, user, function(err, grants) {
      console.log('user successfully saved!');
    });
  });
  console.log('all we need is just a little patience...');
}
\end{verbatim}

Ora il problema � evidente: se i dati ottenuti in modo asincrono sono necessari per eseguire altre procedure, essi vanno gestiti internamente alla callback stessa. Se utilizzate in modo improprio, le callback, portano al cosiddetto \textit{callback-hell}, ovvero un codice nel quale il lettore non riesce pi� a seguire il flusso logico dell'applicazione.

Fortunatamente JavaScript ci fornisce una soluzione semplice al problema, ma che richiede allo sviluppatore disciplina nello scrivere il codice sorgente. L'esempio precedente, infatti potrebbe essere riscritto come segue.

\begin{verbatim}
function onGrantsSaved(err, grants) {
  console.log('user successfully saved!');
}

function onUserSaved(err, user) {
  db.insert(grants, user, onGrantsSaved);
}

function save(user) {
  console.log('saving user into db');
  db.insert(user, onUserSaved);
  console.log('all we need is just a little patience...');
}
\end{verbatim}

Il codice � ora molto pi� leggibile, semplicemente evitando la dichiarazione di funzioni anonime in linea. 


% js nel server
%Questo � un aspetto molto controverso e discusso, infatti se � abituati 
%a pensare tale linguaggio come un "giocattolo", troppo instabile 
%e imprevedibile per 