I prodotti per la gestione centralizzata dei \textit{log} offerti dal mercato sono svariati, ognuno presenta peculiarit� che lo rendono

ce ne sono tanti, alcuni hanno un taglio troppo specifico, ad es. per un linguaggio solo o per un ambiente solo (es. solo mobile)

di seguito ne riportiamo cinque che sono software generici e si adattano ad esigenze diverse e ci hanno ispirato nella realizzazione di mole.io.

\subsubsection{Airbrake}

\begin{figure}[h]
\centering
\includegraphics[width=1.0\linewidth]{./img/airbrake}
\caption[Il sito web di Airbrake]{Il sito web di Airbrake}
\label{fig:airbrake}
\end{figure}

Questa � una delle pi� conosciute applicazioni per il monitoraggio dei log prodotti da applicazioni mobile, in realt� Airbrake offre moduli di integrazione per i principali linguaggi di programmazione e pu� essere utilizzata anche in ambito web o desktop.

A partire dalla versione 2.0 propone un pannello di gestione, ricerca e aggregazione dei messaggi di errore completamente rinnovato. L'interfaccia grafica � gradevole e ben organizzata.

Oltre all'aggregazione automatica dei messaggi d'errore, Airbrake possiede moduli di integrazione con i maggiori sistemi di \textit{bug-tracking}, permettendo di attivare \textit{task} per gli sviluppatori alla ricezione di una segnalazione di errore.

Il formato dei messaggi di errore inviabili a questo \textit{tool} � fisso e non permette l'aggiunta di dati addizionali, questo lo rende abbastanza scomodo quando si vogliano tracciare eventi diversi dagli errori.

\subsubsection{Log.io}

\begin{figure}[h]
\centering
\includegraphics[width=1.0\linewidth]{./img/logio}
\caption[Il sito web di Log.io]{Il sito web di Log.io}
\label{fig:logio}
\end{figure}

La peculiarit� di questo servizio � l'aspetto \textit{realtime}, infatti esso permette di ottenere in tempo reale i log in arrivo dalle applicazioni monitorate.

L'invio dei messaggi al server, da parte delle applicazioni monitorate, avviene tramite messaggi TCP con una formattazione fissa. (Non � proprio user friendly fes...)

I log vengono gestiti come flussi di dati, � possibile applicare alcuni filtri per limitare la quantit� di informazioni riportate ai soli dati utili all'utente. Purtroppo questo sistema non esegue alcun tipo di aggregazione dei dati in arrivo e non permette quindi di avere una visione d'insieme della situazione dell'applicazione monitorata.

\subsubsection{Rollbar}

\begin{figure}[h]
\centering
\includegraphics[width=1.0\linewidth]{./img/rollbar}
\caption[Il sito web di Rollbar]{Il sito web di Rollbar}
\label{fig:rollbar}
\end{figure}

Tra le applicazioni a pagamento censite, questa sembra essere la pi� completa. Permette infatti di registrare sia messaggi di errore sia messaggi generici, esegue una aggregazione relativa ai dati di base di ogni messaggio

da dire: un messaggio ha dati di base (tipo la severity) e dati addizionali tipo messaggi o altri dati strutturati allegabili (stessa idea di mole)
ma la gestione non � proprio pulita, 

\subsubsection{Papertrail}

\begin{figure}[h]
\centering
\includegraphics[width=1.0\linewidth]{./img/papertrail}
\caption[Il sito web di Papertrail]{Il sito web di Papertrail}
\label{fig:papertrail}
\end{figure}

\subsubsection{Fluentd, ElasticSearch e Kibana}

\begin{figure}[h]
\centering
\includegraphics[width=1.0\linewidth]{./img/fluentd}
\caption[Il sito web di Fluentd]{Il sito web di Fluentd}
\label{fig:fluentd}
\end{figure}

\begin{figure}[h]
\centering
\includegraphics[width=1.0\linewidth]{./img/kibana}
\caption[Il sito web di Kibana]{Il sito web di Kibana}
\label{fig:kibana}
\end{figure}















overview di alcuni sistemi di logging con le relative funzioni specifiche
i competitor
airbreak - logga solo
rollbar - aggrega
papertrail - live log

sul mercato esistono svariati sistemi per la gestione centralizzata dei log


Airbrake
+
aggrega log e ne facilita l'analisi
integrato con sistemi di tracking 
-
formati dei messaggi fisso e solo errori


log.io
+
realtime
-
no aggregazione

rollbar
+
non logga solo errori, ma la gestione delle "non eccezioni" � un po' tirata per i capelli
fa aggregazione

https://rollbar.com/vs/airbrake/ questo � mooooolto meglio di mole :(

-
soluzione completa ma blindata e non estensibile (gniiiiiii)



papertrail
+
realtime
aggregazione
-
logga solo tipi di errori noti, no formato flessibile







CACCHIO! fluentd + elasticsearch + kibana = mole
%(mole � pi� facile....gniiiiii)



%mole � pensato pi� come una piattaforma espandibile, accento sulla possibilit� di aggiungere features a caldo