In questo capitolo sar� descritto nel dettaglio Mole.io: un nuovo sistema per la gestione centralizzata dei log, realizzato come progetto di questa tesi.

** Ogni componente del progetto � stato battezzato con un nome di fantasia. Per capire l'idea alla base di questa nomenclatura, � necessaria una introduzione.

\textit{Mole} � una parola inglese che significa \textit{talpa}. Nel gergo dello spionaggio, la talpa, � un infiltrato che viene inserito in un sistema avversario e cattura informazioni che riferisce all'\textit{intelligence} della sua fazione.   

L'applicazione realizzata, si comporta esattamente come un infiltrato: passa informazioni del sistema nel quale viene inserito, cio� le applicazioni da monitorare, alla sua organizzazione: gli sviluppatori.

Ogni talpa che si rispetti ha alcuni contatti all'interno del sistema che gli riportano le informazioni rilevanti. Mole.io possiede agenti con uno scopo simile, chiamati \textit{mole-contact}. Le \textit{soffiate} riferite da ogni mole-contact sono dette \textit{whispers}.

I mole-contact sono moduli software che risiedono all'interno dell'applicazione da monitorare, catturano le situazioni significative per il software nel quale operano ed inviano degli whisper ad un server centrale chiamato \textit{mole}.

Nell'immaginario collettivo, l'infiltrato, � una persona ben vestita, porta un abito elegante, giacca, cravatta e cappello. Anche l'applicazione realizzata per questa tesi, in un certo senso, � ben vestita, infatti, possiede una interfaccia grafica realizzata per monitorare le applicazioni ed organizzare gli whisper in arrivo. Questo componente si � chiamato \textit{mole-suit}.\\

Lo sviluppo dell'applicazione � stato arricchito con uno studio grafico di alcune parti dell'interfaccia, e uno studio del logo. In figura \ref{fig:mole-logo} � riportato il logo proposto per Mole.io.\\

\begin{figure}[h]
\centering
\includegraphics[width=0.7\linewidth]{./img/mole-logo}
\caption[Il logo di Mole.io]{Il logo di Mole.io}
\label{fig:mole-logo}
\end{figure}

Nelle sezioni seguenti verranno illustrate le funzionalit� specifiche di ogni componente in Mole.io, con particolare attenzione alle modalit� con le quali tali sotto-sistemi interagiscono e scambiano dati tra loro.