In questo capitolo descriveremo nel dettaglio Mole.io: un nuovo sistema per la gestione centralizzata dei log, realizzato come progetto di questa tesi.

Iniziamo immediatamente con la doverosa spiegazione dell'origine del nome di questo software.

\textit{Mole} � una parola inglese che significa \textit{talpa}. Nel gergo dello spionaggio, la talpa, � un infiltrato che viene inserito in un sistema avversario e cattura informazioni che riferisce all'\textit{intelligence} della sua fazione.   

Il nostro sistema, si comporta esattamente come un infiltrato: passa informazioni del sistema nel quale viene inserito, le applicazioni da monitorare, alla sua organizzazione, gli sviluppatori.

Ogni talpa che si rispetti ha alcuni contatti all'interno del sistema che gli riportano le informazioni rilevanti. Abbiamo chiamato questi contatti \textit{mole-contacts} e le soffiate da loro riferite \textit{whispers}.

Nel nostro sistema, i \textit{mole-contacts} sono moduli software che risiedono all'interno dell'applicazione da monitorare, catturano le situazioni significative per il software nel quale operano ed inviano degli \textit{whisper} ad un server chiamato \textit{mole}.

Nell'immaginario collettivo l'infiltrato � una persona ben vestita, con un abito elegante, giacca, cravatta e cappello. Anche il nostro software, in un certo senso, � ben vestito, infatti possiede una interfaccia grafica realizzata per monitorare le applicazioni ed organizzare gli whisper in arrivo. Questo componente si chiama \textit{mole-suit}.\\

Il progetto al momento � un prototipo, ma il � destinato ad essere ultimato per diventare un servizio vero e proprio fornito via web. in figura \ref{fig:mole-logo} � riportato il logo proposto per Mole.io.\\

\begin{figure}[h]
\centering
\includegraphics[width=0.7\linewidth]{./img/mole-logo}
\caption[Il logo di Mole.io]{Il logo di Mole.io}
\label{fig:mole-logo}
\end{figure}

Nella sezione seguente vedremo quali sono le funzionalit� specifiche di ogni componente in Mole.io, con particolare attenzione al modo nel quale tali componenti scambiano dati tra loro.

