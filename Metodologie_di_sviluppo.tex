Per lo sviluppo del progetto di tesi, sono state adottate tecniche di progettazione e sviluppo del software che provengono dall'ambito delle metodologie di sviluppo agili.

Il modello di sviluppo agile � un insieme di pratiche basate sulla costruzione iterativa e incrementale del software. 
Il flusso di lavoro � organizzato in cicli di breve durata, che hanno come oggetto l'implementazione di una piccola quantit� di \textit{feature}. Questa pratica di sviluppo iterativo, offre un immediato vantaggio: la possibilit� di riesaminare il lavoro effettuato al ciclo precedente e decidere, a seconda delle esigenze correnti, se continuare lo sviluppo nella stessa direzione o cambiare radicalmente approccio.

Questa possibilit� � fondamentale nell'ottica della buona riuscita di un progetto, poich� � molto frequente che, durante lo sviluppo di applicativi software, i committenti decidano, per vari motivi, di sostituire logiche di funzionamento del software. 
Lo sviluppo iterativo, quindi, mette in condizione il team di lavoro di proporre soluzioni che si adattano nel tempo alle specifiche, generando cos� un prodotto strettamente aderente alle aspettative del cliente.

Tra le tecniche operative pi� frequenti adottate nello sviluppo di un progetto, citiamo:
\begin{itemize}
\item pianificazione adattiva
\item sviluppo evolutivo
\item rilasci frequenti
\item \textit{time-boxing}
\end{itemize}
Lo scopo di queste tecniche � ottenere un modello di sviluppo che possa essere flessibile e adattarsi bene al cambiamento. 

\subsubsection{Lo Sviluppo Iterativo e Incrementale}

Il processo di sviluppo � suddiviso in unit� base, con una durata temporale che va da una a quattro settimane.
Conseguentemente non � possibile organizzare le attivit� di sviluppo tenendo conto dei soli obiettivi globali del progetto, ma � necessario ri-contestualizzarli rispetto alla finestra temporale adottata. Gli obiettivi globali vengono cos� suddivisi in task pi� piccoli, ciascuno focalizzato su una problematica specifica. 

Al termine di ogni unit� base, si procede con la successiva, in un processo iterativo. Il risultato di una iterazione dovrebbe essere un prodotto parzialmente funzionante da mostrare al cliente.

Questo approccio minimizza il rischio di portare lo sviluppo "fuori dal contesto" e permette di avvicinare il cliente al processo di realizzazione del software, rendendolo partecipe dei problemi incontrati e permettendogli di ottenere un prodotto molto aderente alle proprie esigenze. 

\subsubsection{La Comunicazione Rapida}
Le metodologie agili suggeriscono una organizzazione gerarchica, con un ristretto numero di livelli, dei team di sviluppo. Ogni team elegge il proprio \textit{owner} che � responsabile del gruppo di lavoro ed � l'unica interfaccia con i superiori. Anche in questo caso l'esigenza � creare una catena di comunicazione il pi� corta possibile, in modo da permettere agli sviluppatori di ottenere, in brevissimo tempo, un \textit{feedback} riguardo ai problemi incontrati. 

Un altro strumento utile a questo scopo sono gli \textit{stand-up meeting}: riunioni giornaliere brevissime, nelle quali, per salvaguardare la brevit� dell'incontro, i partecipanti formano un cerchio stando in piedi. In questi incontri ogni sviluppatore riporta all'\textit{owner} l'elenco dei lavori sul quale � impegnato, stime di tempi per la chiusura delle \textit{feature} in sviluppo, eventuali problematiche incontrate e programmazione delle attivit� di sviluppo nell'immediato futuro.

\subsubsection{Mantenere Alta la Qualit�} 
Le metodologie agili prevedono l'impiego di strumenti software avanzati per supportare le tecniche descritte e facilitare il raggiungimento degli obiettivi prefissati. In \cite{Martin:2008:CCH:1388398} e \cite{Martin:2011:CCC:1999258} Robert C. Martin descrive alcuni strumenti e varie tecniche utilizzate per perseguire l'obiettivo della qualit� del software. Nella trattazione, tra gli altri, sono annoverati:
\begin{itemize}
 \item \textit{continuous-integration}
 \item test automatici 
 \item \textit{pair programming}
 \item \textit{test-driven development} (TDD) 
 \item \textit{design patterns} \cite{Freeman:2004:HFD:1076324}
 \item \textit{user stories}
\end{itemize}

Nelle sezioni seguenti approfondiremo alcune delle tecniche utilizzate durante la progettazione e lo sviluppo del progetto.