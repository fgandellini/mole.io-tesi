Iniziamo riportando alcune definizioni che utilizzeremo spesso nel seguito della tesi. Diciamo \textit{processo} un programma in esecuzione e \textit{log} l'insieme dei messaggi prodotti, a fini informativi, da tale processo.

La decisione di quali e quante informazioni salvare, spetta tipicamente allo sviluppatore o al personale addetto alla gestione dell'applicazione.

I messaggi di log posso essere di vario tipo, � usanza comune caratterizzare ogni messaggio con un livello di gravit� (\textit{severity}) permettendo cos� una identificazione pi� rapida degli errori pi� gravi, rendendo repentino l'intervento di riparazione dell'applicazione.

Informazioni spesso salvate all'interno dei log sono dati specifici del sistema nel quale l'applicazione � in esecuzione, dati relativi all'utente che la sta utilizzando, oppure relativi allo stato del sistema in un preciso istante temporale. Ogni log � infine corredato da un messaggio significativo che lo rende immediatamente identificabile tra altri.



