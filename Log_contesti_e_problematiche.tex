Prima di entrare nello specifico delle tematiche trattate, � necessario riprendere e chiarire alcuni concetti chiave che utilizzeremo ripetutamente nel seguito della tesi.

Si definisce \textit{processo} un programma in esecuzione e \textit{log} l'insieme dei messaggi prodotti, a fini informativi, da tale processo.

I messaggi di log posso essere di vario tipo. \`{E} usanza comune caratterizzare ogni messaggio con un livello di gravit� (\textit{severity}) permettendo cos� una rapida identificazione degli errori critici allo scopo di rendere repentino l'intervento di riparazione dell'applicazione.

Bench� esista un protocollo riconosciuto a livello internazionale e adottato negli ambienti \textit{Unix-like} chiamato \textit{SysLog}, nell'ambito dello sviluppo di applicativi, non esistono norme vincolanti per la strutturazione dei log stessi. Questo accade sia perch� SysLog non rappresenta uno standard rigidamente definito, sia perch� l'organizzazione delle informazioni contenute nei log, � spesso delegata agli sviluppatori, i quali implementano questa funzionalit� nel modo pi� conveniente rispetto alle specifiche esigenze dell'applicazione in costruzione.  

All'interno dei log, di conseguenza, troveremo informazioni diversificate in base al caso d'uso, ma tra le pi� frequenti ci sono:
\begin{itemize}
\item dati specifici del sistema nel quale l'applicazione � in esecuzione;
\item dati relativi all'utente che sta utilizzando l'applicazione;
\item dati relativi allo stato del sistema in un preciso istante temporale;
\item un marcatore temporale (\textit{timestamp})
\item un messaggio in linguaggio naturale, significativo, che lo rende immediatamente identificabile tra altri;
\end{itemize}
