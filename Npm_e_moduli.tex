L'organizzazione del codice � un aspetto fondamentale dello sviluppo. Il \textit{Single Responsibility Principle} (SRP) � uno dei principi di design del software pi� importanti. Applicato alla programmazione ad oggetti, esso sostiene che ogni oggetto deve essere responsabile di un singolo aspetto del comportamento del sistema. 

Node.js incarna questo principio nelle fondamenta della sua struttura, infatti permette di organizzare il codice in unit� indipendenti tra loro chiamate \textit{moduli}. Ogni modulo nella piattaforma pu� decidere quali dati gestire al suo interno e quali esporre all'esterno. Si creano in questo modo delle \textit{black box} che funzionano tanto meglio quanto il loro compito � specifico. 

Poco dopo la nascita di Node.js la comunit� di sviluppatori che lo utilizzava ha deciso di arricchire questa piattaforma con un \textit{tool} ormai insostituibile: \textit{Node Package Manager} (NPM).

%TODO: logo NPM (?)

NPM � un tool, utilizzabile da linea di comando, che si occupa della gestione dei moduli e delle loro dipendenze, per farlo utilizza un \textit{repository} in rete nel quale sono registrati tutti i moduli pubblici sviluppati dai vari \textit{contributor} della \textit{community} Node.js.

Abbiamo introdotto il concetto di \textit{dipendenza} tra moduli. Un modulo si dice dipendente da un altro quando necessita di quest'ultimo per eseguire il suo compito. NPM impone che ogni modulo sia descritto dal file \verb|Package.json| che ne riporta le informazioni principali ed elenca gli altri moduli dai quali dipende. Di seguito riportiamo un esempio di \verb|Package.json| per un modulo chiamato \verb|mole| che fa parte del sistema realizzato per questa tesi.

\begin{verbatim}
{
  "name": "mole",
  "version": "0.0.1",
  "author": "Federico Gandellini",
  "description": "whisper collector and denormalizer",
  "private": true,
  "scripts": {
    "start": "node mole.js"
  },
  "engines": {
    "node": ">=0.8.0",
    "npm": ">=1.2.0"
  },
  "dependencies": {
    "express": "3.3.4",
    "underscore": "~1.5.2",
    "mongo-make-url": "0.0.1",
    "mongoskin": "~0.6.0",
    "mongodb": "~1.3.19",
    "require-all": "0.0.8",
    "rabbit.js": "~0.3.1"
  },
  "devDependencies": {
    "grunt-contrib-jshint": "~0.6.4",
    "grunt": "~0.4.1",
    "grunt-mocha-cli": "~1.2.1",
    "grunt-contrib-watch": "~0.5.3",
    "mocha": "~1.13.0",
    "should": "~1.3.0",
    "supertest": "~0.8.0",
    "Faker": "~0.5.11"
  }
}
\end{verbatim}

Nella prima parte del file possiamo trovare i dati principali del modulo, come il suo nome, l'autore, la versione, e una descrizione. Seguono un paio di parametri che definiscono le regole di pubblicazione, il comando per lanciare questo modulo e le versioni richieste della piattaforma Node.js e di NPM.

Le due sezioni seguenti nel file elencano le dipendenze, la prima indica i moduli che devono essere presenti perch� esso possa essere messo \textit{in produzione}, le altre sono dipendenze che sono richieste esclusivamente durante lo sviluppo o il \textit{testing} del modulo stesso. Come si pu� notare, nel file, non sono presenti solo i nomi, ma anche le versioni richieste degli altri moduli.

Uno dei comandi pi� utlizzati di NPM � \verb|npm install|, esso legge il file \verb|Package.json| presente nella \textit{directory} corrente e scarica dalla rete tutti i pacchetti richiesti alle rispettive versioni e permette all'utilizzatore del modulo di essere immediatamente operativo.

\subsubsection{Alcuni moduli utilizzati}

Per poter utilizzare le funzionalit� fornite da un modulo aggiuntivo, Node.js fornisce un comando che permette di importarlo dall'esterno. Questo comando � \verb|require()| e si utilizza passando come parametro il nome del modulo da caricare. 

\begin{verbatim}
var express = require('express');
\end{verbatim}

Una volta eseguito il comando, la variabile \verb|express| conterr� il modulo appena caricato e sar� possibile utilizzarne le funzionalit�.\\

Illustriamo ora alcuni dei moduli utilizzati per realizzare la nostra applicazione.

\begin{description}
\item [express] \`{E} uno dei moduli pi� importanti: permette di creare facilmente un \textit{server web} in grado di rispondere a richieste provenienti dai \textit{client}. Utilizza un sistema di \textit{rotte} per legare l'azione da eseguire alla richiesta HTTP in arrivo. Express eredita da \textit{Connect} una funzionalit� chiave per il suo funzionamento, i \textit{middleware}. In Connect, un middleware � una funzione che filtra tutte le richieste in ingresso e le risposte in uscita e le restituisce rielaborate. I middleware sono costruiti in modo da essere accodati l'uno con l'altro, dando la possibilit� di costruire filtri molto complessi. Illustreremo la configurazione delle rotte e dei middleware di express nella sezione \ref{Architettura_del_sistema}.

\item [passport] Il compito di questo modulo � gestire l'autenticazione degli utenti. Passport fornisce un comodo middleware per express, con il quale � possibile verificare i dati di autenticazione dell'utente quando questo voglia accedere a specifiche rotte. Pi� avanti, nella sezione \ref{Autenticazione_degli_utenti}, approfondiremo questo tema.

\item [mongoose e mongoskin] Sono i due principali \textit{driver} Node.js per MongoDB, il sistema per l'archiviazione dei dati che abbiamo utilizzato nell'applcazione. Nella sezione \ref{MongoDB} vedremo nel dettaglio questo database documentale e illustreremo come � possibile accedere alla sua interfaccia utilizzando Node.js.

\item [require-all] Permette di caricare tutti i moduli presenti in una directory. Questo modulo ci � stato molto utile per garantire e semplificare l'estensibilit� del sistema. Nella sezione \ref{CQRS_ed_estensibilita} vedremo nel dettaglio come abbiamo sfruttato questa semplice ma potente funzionalit�.

\item [mocha] Questo modulo non fornisce funzionalit� vere e proprie da spendere in produzione, bens� un insieme preziosissimo di \textit{tool} per poter testare la propria applicazione Node.js. Abbiamo largamente utilizzato questo strumento durante la realizzazione del sistema per renderlo, quanto pi� possibile, \textit{bug-free}. %TODO: qui andrebbe linkata la sezione sul testing...ma � prima!
\end{description}


