%TODO
% workflow con webhooks
% notifiche push (si potrebbe usare un denormalizer?)
% app mobile per riceverle
% https
% marcatura degli errori come "gestito" e conseguente archiviazione
% integrazione con sistemi di tracking (?)
% aggiungere pi� mole-contacts
% aggiungere un sistema di query
% mole-contacts -> caching degli errori in locale quando non c'� rete (buffering)

% migliorare la gestione dgli widget e plugin per l'aggiunta a caldo

% possibili nuovi denormalizzatori

% bench e vm sulla stessa macchina
% vm con cpu pin
% ramp up








-- ABSTRACT --

Mole.io � un sistema centralizzato per la raccolta e l'aggregazione di messaggi provenienti da applicazioni remote.

L'approccio comune alla creazione e gestione dei log presenta la criticit� specifica della \textit{localit�}, poich� abitualmente i file di log vengono salvati nella stessa macchina sulla quale sta operando l'applicazione che li ha prodotti.

All'aumentare del numero di applicazioni da gestire e del numero di macchine in produzione, diventa sempre pi� difficoltoso ottenere un \textit{feedback} veloce dello stato di ogni software e delle eventuali situazioni di errore in cui le applicazioni si trovano.

Mole.io cerca di risolvere il problema facendo in modo che i software che lo utilizzano, siano in grado di inviare le informazioni che ritengono significative ad un server centrale, il quale le raccoglie, le cataloga e le aggrega per essere facilmente supervisionate da parte degli sviluppatori.

Il nuovo approccio permette agli sviluppatori di identificare, in breve tempo, il manifestarsi di un malfunzionamento in qualunque applicazione attiva e reagire rapidamente proponendo una azione risolutiva.

Mole.io � stato realizzato con l'obiettivo di risolvere la problematica di gestione dei log, garantendo comunque un sistema flessibile  

% un'occhio all'estensibilit�

Per il \textit{deploy} in produzione di Mole.io, si � scelto di utilizzare \textit{Microsoft Azure}, un sistema \textit{PaaS} distribuito con il quale � possibile creare architetture facilmente scalabili per supportare la variabilit� del carico di lavoro richiesto al sistema. 
