Il salvataggio dei log, come abbiamo anticipato, � una operazione molto comune nei software, ma diventa fondamentale quando si vuole monitorare lo stato interno di una applicazione in esecuzione e si vuole essere informati riguardo alle situazioni di errore nelle quali quest'ultima incorre.

Salvare le informazioni relative alle situazioni di errore � importante per gli sviluppatori, questo permette, infatti di velocizzare l'individuazione di errori (\textit{bug}) nel flusso di lavoro del programma e, di conseguenza la loro risoluzione (\textit{bugfix}).

Il salvataggio dei log avviene tipicamente su uno o pi� \textit{files} di testo presenti nella stessa macchina nella quale sta funzionando l'applicazione. A seconda del tempo di esecuzione di una applicazione e della frequenza con la quale essa produce messaggi di log, i files sui quali vengono salvate le informazioni, possono diventare molto grandi. File di considerevoli dimensioni, sono altamente complessi da gestire da parte degli addetti ai lavori, infatti trovare informazioni significative all'interno di questa grande mole di dati � un processo che richiede molto tempo.

Il problema dei file di grandi dimensioni non riguarda esclusivamente la scansione sequenziale delle informazioni, bens� l'individuazione del messaggio prodotto a fronte di una criticit� e la correlazione di quest'ultima allo stato del sistema nell'istante in cui � stata generata.

Il tempo che intercorre tra la scoperta di un bug e il relativo bugfix dovrebbe essere il pi� possibile contenuto. Riuscire a correlare le situazioni di errore tra loro e capire quali condizioni del sistema generano un malfunzionamento, dovrebbe essere una operazione immediata. Spesso, purtroppo, le eccessive dimensioni dei file di log rendono questa procedura molto costosa.\\

%Se caliamo questa problematica nell'ambito delle applicazioni web

%La difficolt� principale non � scandire file grandi, bens� trovare rapidamente uno specifico errore 

%diventa lungo e complesso trovare informazioni significative all'interno di esso e soprattutto diventa complesso correlare le situazioni di errore e capire con quale frequenza o in quali condizioni si presenta un particolare malfunzionamento.

problema localit�: se ho molte applicazioni e molti clienti devo tener monitorata (da remoto) la situazione dei log per ogni cliente/app/macchina, verificare che non esplodano i files e quando diventano troppo grandi, archiviare parte di questi. 



problema spazio, 
localit�
difficili da leggere e difficile tirarci fuori delle info significative 