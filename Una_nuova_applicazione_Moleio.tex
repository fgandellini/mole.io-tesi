L'analisi dello stato dell'arte delle tecnologie presenti sul mercato, ha portato CodicePlastico a perseguire la scelta dello sviluppo di una nuova applicazione realizzata internamente all'azienda stessa.

Le motivazioni che hanno determinato questa scelta, sono molteplici:
\begin{description}
\item[Business] la maggior parte delle soluzioni analizzate � fornita come \textit{Software as a Service} (SaaS). In questo caso sarebbe d'obbligo far adottare tale soluzione ad ogni cliente, perdendo di fatto, il vantaggio di avere un sistema centralizzato di raccolta dei dati.
\item[Riservatezza] le soluzioni analizzate che sono fornite come SaaS, prevedono che i messaggi vengano salvati su basi di dati remote e non di propriet� dell'azienda. Questo espone potenzialmente a problematiche di gestione della riservatezza dei dati sensibili che sarebbero pi� facilmente controllabili utilizzando un sistema \textit{self-hosted}.
\item[Personalizzazione] a parte l'ultima soluzione analizzata (Fluentd, Elastic Search e Kibana) la maggior parte dei software presenta difficolt� nella personalizzazione del contenuto dei messaggi, fondamentale per l'azienda, sia per tracciare con precisione problematiche specifiche delle applicazioni in produzione, sia nell'ottica della raccolta dei dati per la Business Intelligence.
\item[Manutenzione] limitare gli interventi di manutenzione ad un'unica applicazione proprietaria, di cui, conseguentemente, si ha pieno controllo dello sviluppo.
\item[Estensibilit�] l'intero sviluppo delle varie componenti dell'applicazione � stato guidato dal concetto di estensibilit�. Il sistema risultante dovr� essere facilmente estendibile con nuove funzionalit� e \textit{feature} per adattarsi nel modo pi� aderente possibile alle esigenze di ogni cliente. Un ulteriore aspetto che ha motivato alla costruzione di una nuova applicazione, � la possibilit� di rilasciare gradualmente agli utenti le nuove funzionalit� introdotte, man mano, nel sistema (\textit{deploy} graduale).
\item[Sperimentazione] tra i valori aziendali fondamentali di CodicePlastico vengono annoverati l'aggiornamento costante del personale e il miglioramento delle \textit{skill} di ogni sviluppatore, anche attraverso pratiche agili come \textit{pair programming}, \textit{randori} e progetti personali. Lo sviluppo di una applicazione con l'ausilio di tecnologie relativamente recenti e innovative nel panorama IT, � in completo accordo con la \textit{vision} aziendale.
\end{description}