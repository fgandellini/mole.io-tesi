Il salvataggio dei dati � una operazione critica in moltissimi sistemi software. La scelta di quale database utilizzare per salvare le informazioni � altrettanto delicata e va ponderata alla luce di vari punti di vista. Per la nostra applicazione abbiamo scelto di utilizzare MongoDB. Di seguito cercheremo di illustrare le principali caratteristiche di questo database e le motivazioni che ci hanno spinto a sceglierlo per la nostra applicazione.\\

\begin{figure}[h]
\centering
\includegraphics[width=0.7\linewidth]{./img/mongodb}
\caption[Il logo di MongoDB]{Il logo di MongoDB}
\label{fig:mongodb}
\end{figure}

I database relazionali (RDBMS), per loro natura, espongono il loro contenuto in un formato tabellare e utilizzano concetti come righe e colonne per fornire l'accesso ai dati o porzioni di essi. In questo tipo di database lo \textit{schema} dei dati, cio� la loro struttura � ben definita e va studiata in fase di \textit{design} della base di dati, essi si definiscono infatti database \textit{schema-full}. Negli RDBMS la modifica del formato di un dato a sistema avviato � una operazione delicata. Essa infatti deve tenere conto dei dati gi� presenti all'interno del sistema e deve aggiornarli coerentemente con la nuova struttura assunta dalla tabella che li contiene.

MongoDB � un database \textit{schema-less}, questo significa che la struttura di un dato non � definita a priori, ma soprattutto che essa pu� variare nel tempo senza richiedere aggiornamenti ai dati gi� presenti nella base dati. MongoDB utilizza infatti un modello documentale per il salvataggio dei dati, essi salvati internamente come oggetti \textit{BSON} e forniti all'esterno sotto forma di oggetti \textit{JSON}.   

Per comprendere i concetti alla base di MongoDB � possibile realizzare alcune similitudini tra concetti proposti da questa base dati e concetti presenti negli RDBMS. La tabella seguente mostra alcune di queste similitudini.

\begin{center}
\begin{tabular}{l|l}
\textbf{RDBMS} & \textbf{MongoDB} \\ 
\hline 
table & collection \\ 
tuple & document \\ 
column & field \\
\end{tabular} 
\end{center}

Le \textit{collection} sono insiemi di \textit{document}, i quali, a loro volta, contengono vari \textit{field} che rappresentano le chiavi per l'accesso ai dati veri e propri: i \textit{value}.

Un documento BSON pu� contenere \textit{field} di vario tipo: interi, stringhe, \textit{array}, dati binari, oppure altri documenti (\textit{embedded document}). Come abbiamo anticipato, in MongoDB, lo schema dei documenti non � fisso, questo significa che nella stessa collection potremo trovare document con struttura differente.

Altre funzionalit� significative di MongoDB sono la possibilit� di eseguire aggiornamenti atomici dei dati, la ricerca \textit{full-text} all'interno dei documenti e degli \textit{embedded document}, la possibilit� di creare indici sui dati e indici di tipo geospaziale.

La possibilit� di salvare dati aventi una struttura variabile � stato il motivo principale che ci ha spinto ad utilizzare questo tipo di database, nel capitolo \ref{mole} vedremo come questa funzionalit� ci ha permesso di creare un sistema estremamente flessibile.

I creatori di MongoDB hanno fatto in modo che il database da loro realizzato realizzasse due caratteristiche fondamentali, che lo rendono il candidato ideale per l'installazione in ambienti PaaS: l'alta accessibilit� e la scalabilit�.

Nella sezione seguente vedremo come MongoDB implementa questi due concetti e come essi si posso utilizzare sia per fronteggiare l'aumento di richieste da parte degli utenti, sia per rendere il sistema resistente a problemi di malfunzionamento dei server sui quali MongoDB viene installato.

Nella sezione \ref{Architettura_del_sistema} vedremo inoltre come queste caratteristiche hanno reso MongoDB il candidato ideale ad essere utilizzato all'interno del nostro sistema. Nella configurazione finale, infatti, esso verr� installato proprio su un servizio di tipo PaaS.
