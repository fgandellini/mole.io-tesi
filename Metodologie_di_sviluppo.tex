Per lo sviluppo del progetto di tesi, ci siamo affidati a tecniche di progettazione e sviluppo del software che provengono dal mondo dei metodi di sviluppo agile.

(non abbiamo imparato solo nuovi sw, ma anche nuove metodologie...)


Lo il modello di sviluppo agile � un insieme di metodologie basate sulla costruzione iterativa e incrementale del software. 

Lo sviluppo iterativo permette infatti di riesaminare il lavoro effettuato al ciclo precedente e decidere, a seconda delle esigenze correnti, se continuare lo sviluppo nella stessa direzione o cambiare radicalmente approccio.

-- E' infatti molto frequente che durante lo sviluppo di applicativi software i committenti decidano, per vari motivi, di sostituire logiche di funzionamento del progetto.

L'obiettivo � raggiunto attraverso tecniche come pianificazione adattativa, sviluppo evolutivo, rilasci frequenti e \textit{time-boxing} delle attivit�. Lo scopo di queste tecniche � ottenere un modello di sviluppo che possa essere flessibile e adattarsi bene al cambiamento. 

\begin{description}
\item[Sviluppo iterativo e incrementale] Il processo di sviluppo � suddiviso in unit� base con una durata temporale che va da una a quattro settimane. Gli obiettivi globali di progetto vengono, di conseguenza, sostituiti da obiettivi relativi alla nuova finestra temporale, garantendo una gestione del tempo migliore e una maggior \textit{focus} su problemi specifici. Al termine di una unit� base, si procede con la successiva, in un processo iterativo. Il risultato di una iterazione dovrebbe essere un prodotto parzialmente funzionante da mostrare al cliente. Questo approccio minimizza il rischio di portare lo sviluppo "fuori dal contesto" e permette di avvicinare il cliente al processo di realizzazione del software, rendendolo partecipe dei problemi incontrati e permettendogli di ottenere un prodotto molto aderente alle proprie esigenze. 
\item[Comunicazione rapida] La metodologie agili suggeriscono una organizzazione dei team di sviluppo gerarchica, nella quale ogni team elegge il proprio \textit{owner}, il quale � responsabile del team e riferisce ed � l'unica interfaccia con i superiori. Anche in questo caso l'esigenza � creare una catena di comunicazione il pi� corta possibile, in modo da permettere agli sviluppatori di ottenere un \textit{feedback} riguardo ai problemi incontrati, in brevissimo tempo. An altro strumento utile a questo scopo sono gli \textit{stand-up meeting}: riunioni giornaliere brevissime nelle quali ogni sviluppatore riporta all'\textit{owner} l'elenco dei lavori sul quale � impegnato, stime di tempi per la chiusura delle \textit{feature} in sviluppo, eventuali problematiche incontrate e [planning per l'immadiato futuro]
\item[Mantenere alta la qualit�] le metodologie agili prevedono l'impiego di strumenti software avanzati per supportare le tecniche descritte e facilitare il raggiungimento degli obiettivi prefissati, sempre con l'obiettivo di fornire al cliente un software il pi� vicino possibile alle sue esigenze, quindi con un valore percepito molto alto. Gli strumenti e le tecniche utilizzate comprendono \textit{continuous-integration}, test automatici, \textit{pair programming}, \textit{test-driven development} (TDD), \textit{design patterns}, \textit{user stories} e molto altro.
\end{description}

Nelle sezioni seguenti approfondiremo alcune delle tecniche utilizzate durante la progettazione e lo sviluppo del progetto.

% clean code
\cite{Martin:2008:CCH:1388398}

% the clean coder
\cite{Martin:2011:CCC:1999258}

%Agile software development is a group of software development methods based on iterative and incremental development, where requirements and solutions evolve through collaboration between self-organizing, cross-functional teams. It promotes adaptive planning, evolutionary development and delivery, a time-boxed iterative approach, and encourages rapid and flexible response to change. It is a conceptual framework that promotes foreseen tight interactions throughout the development cycle.
%The Agile Manifesto[1] introduced the term in 2001. Since then, the Agile Movement, with all its values, principles, methods, practices, tools, champions and practitioners, philosophies and cultures, has significantly changed the landscape of the modern software engineering and commercial software development in the Internet era.

%There are many specific agile development methods. Most promote development, teamwork, collaboration, and process adaptability throughout the life-cycle of the project.

%Iterative, incremental and evolutionary
%Agile methods break tasks into small increments with minimal planning and do not directly involve long-term planning. Iterations are short time frames (timeboxes) that typically last from one to four weeks. Each iteration involves a cross-functional team working in all functions: planning, requirements analysis, design, coding, unit testing, and acceptance testing. At the end of the iteration a working product is demonstrated to stakeholders. This minimizes overall risk and allows the project to adapt to changes quickly. An iteration might not add enough functionality to warrant a market release, but the goal is to have an available release (with minimal bugs) at the end of each iteration.[10] Multiple iterations might be required to release a product or new features.

%Efficient and face-to-face communication
%No matter what development disciplines are required, each agile team will contain a customer representative, e.g. Product Owner in Scrum. This person is appointed by stakeholders to act on their behalf[11] and makes a personal commitment to being available for developers to answer mid-iteration questions. At the end of each iteration, stakeholders and the customer representative review progress and re-evaluate priorities with a view to optimizing the return on investment (ROI) and ensuring alignment with customer needs and company goals.
%In agile software development, an information radiator is a (normally large) physical display located prominently in an office, where passers-by can see it. It presents an up-to-date summary of the status of a software project or other product.[12][13] The name was coined by Alistair Cockburn, and described in his 2002 book Agile Software Development.[13] A build light indicator may be used to inform a team about the current status of their project.

%Very short feedback loop and adaptation cycle
%A common characteristic of agile development are daily status meetings or "stand-ups", e.g. Daily Scrum (Meeting). In a brief session, team members report to each other what they did the previous day, what they intend to do today, and what their roadblocks are.

%Quality focus
%Specific tools and techniques, such as continuous integration, automated unit testing, pair programming, test-driven development, design patterns, domain-driven design, code refactoring and other techniques are often used to improve quality and enhance project agility.