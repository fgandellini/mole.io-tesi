In questa tesi descriveremo Mole.io: un sistema centralizzato per la raccolta e l'aggregazione di messaggi provenienti da applicazioni remote.

Durante il loro ciclo di lavoro o \textit{processing}, le applicazioni software eseguono operazioni significative o entrano in situazioni di errore, in questi casi � importante che le persone cha hanno in carico la gestioni di questi sistemi, siano informate dell'accaduto in modo da operare scelte opportune o applicare le dovute correzioni (\textit{bugfix}).

Gli sviluppatori spesso utilizzano messaggi di tracciamento (\textit{log}) per stampare a video o salvare in \textit{files} stati significativi delle applicazioni. Gli stessi \textit{log} sono utilizzati pi� spesso per riportare situazioni di errore (\textit{Exception} e \textit{Stack Trace}).

Il problema principale di questo approccio � la \textit{localit�} dei \textit{log}, solitamente questi \textit{files} vengono salvati, nella stessa macchina sulla quale sta operando l'applicazione.

All'aumentare del numero di applicazioni da gestire e del numero di macchine in produzione, capita spesso che i server siano in luoghi geograficamente distanti tra loro. Questa situazione rende evidente la difficolt� di ottenere un feedback veloce dello stato di ogni software e delle eventuali situazioni di errore in cui le applicazioni si trovano.

Mole.io cerca di risolvere il problema facendo in modo che i software che lo utilizzano, siano in grado di inviare le informazioni che ritengono significative ad un server centrale, che le raccoglie, le cataloga e le aggrega per essere facilmente supervisionate da parte degli sviluppatori.

Nel primo capitolo tratteremo approfonditamente il problema dei \textit{log}, i contesti nei quali essi vengono utilizzati e le problematiche legate alla gestione di questo tipo di soluzione di tracciamento. Vedremo anche come utilizzare i \textit{log} per ottenere informazioni di supporto alla \textit{business intelligence}. 

Il secondo capitolo riporter� un elenco dei principali \textit{software} per la gestione centralizzata dei \textit{log} presenti sul mercato e delle soluzioni \textit{Open Source} che sono state prese a modello per la realizzazione di Mole.io. Descriveremo ogni applicazione e mostreremo come Mole.io possa essere una soluzione innovativa sotto svariati punti di vista.

I due capitoli seguenti permetteranno di approfondire i dettagli tecnici delle metodologie di sviluppo applicate durante il \textit{design} del software e alcune tra le principali tecnologie utilizzate per la realizzazione del sistema.

Il quinto capitolo descriver� la struttura di Mole.io e le varie componenti software che rendono l'applicazione scalabile, sicura e garantiscono l'alta affidabilit� della soluzione. Uno spazio particolare sar� inoltre riservato alle problematiche incontrate durante lo sviluppo.

Nel sesto capitolo vedremo in modo oggettivo, con \textit{benchmark} e \textit{stress test} il comportamento di Mole.io all'aumentare del carico di lavoro e dimostreremo come le soluzioni di design applicate garantiscano buone \textit{performance} anche in condizioni critiche.

Infine discuteremo i risultati ottenuti e proporremo alcune interessanti funzionalit� che trasformeranno Mole.io dall'attuale \textit{proof of concept} ad un vero e proprio servizio.
