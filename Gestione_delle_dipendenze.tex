Per una azienda o uno sviluppatore � molto dispendioso, in termini di tempo e denaro, realizzare le diverse funzionalit� implementandole in modo completamente autonomo. Per questo motivo � comune l'utilizzo di librerie che realizzano sotto-funzionalit� e possono essere incluse nelle applicazioni come \textit{black-box}.

Gli sviluppatori di frontend, oggi, hanno a loro disposizione un panorama vastissimo di librerie per la gestione delle pi� disparate problematiche legate alla realizzazione delle funzionalit� client-side: gallerie di immagini, strumenti per la navigazione, maschere per l'inserimento di dati da parte dell'utente e molto altro.

Il processo che porta all'integrazione di una libreria esterna nel proprio progetto, si articola in diverse fasi:
\begin{itemize}
\item ricerca della libreria appropriata;
\item verifica della compatibilit� e degli eventuali problemi di integrazione con altre librerie gi� in uso;
\item inclusione nei diversi file dell'applicazione;
\item utilizzo delle funzionalit� fornite;
\end{itemize}

Questa procedura, si ripete per tutte le librerie incluse nell'applicazione. Si pu� immaginare come esso sia facilmente automatizzabile, nell'ottica di velocizzare il processo di sviluppo e assicurare una solida integrazione tra i sistemi.

La costruzione del frontend risente di problematiche simili a quella del backend e di soluzioni altrettanto simili. Come illustrato nel capitolo \ref{Npm_e_moduli}, l'utilizzo di tool di gestione delle dipendenze aiuta a evitare eventuali errori ascrivibili al fattore umano, durante integrazione di librerie di terze parti, inoltre permette di velocizzare tali operazioni.

Yeoman � un insieme di strumenti per lo sviluppo di applicazioni client-side. Nasce con l'obiettivo di fornire allo sviluppatore un framework per svolgere il proprio lavoro in modo semplice e veloce, ma, al tempo stesso, produrre applicazioni web di alta qualit�.\\

\begin{figure}[h]
\centering
\includegraphics[width=0.7\linewidth]{./img/yo-grunt-bower.png}
\caption[I loghi di Yo, Grunt e Bower]{I loghi di Yo, Grunt e Bower}
\label{fig:yo-grunt-bower}
\end{figure}

Yeoman � un tool che si utilizza da linea di comando, offre molte funzionalit� per l'automazione dei diversi task e per il supporto allo sviluppo, di seguito le principali:
\begin{itemize}
\item generazione automatica di \textit{template};
\item gestione delle dipendenze delle diverse librerie incluse nel progetto;
\item esecuzione automatica di test unitari;
\item gestione di un server da utilizzare durante la fase di sviluppo;
\item ottimizzazione del codice realizzato al fine di eseguire il deploy del progetto.
\end{itemize}

Questo strumento combina diversi software open-source che si occupano di fornire le varie funzionalit� offerte. Utilizza un concetto di \textit{generator}: un processo automatico per la produzione di una applicazione, preconfigurata con varie librerie di supporto. Yeoman, inoltre, dispone di una modalit� interattiva, nella quale guida lo sviluppatore nella scelta delle diverse librerie da includere nel progetto generato.

I software che concorrono a rendere Yeoman un tool potente e versatile sono:
\begin{description}
\item[Yo] il vero e proprio sistema di generazione delle applicazioni. Si occupa di scrivere i diversi file di configurazione degli altri tool e includere le dipendenze necessarie per lo sviluppo del progetto.
\item[Grunt] � un software di \textit{task automation}. Il suo compito � l'automatizzazione di compiti ripetitivi. \`{E} sviluppato in Node.js e permette di definire, usando un file di configurazione in formato JSON, la sequenza delle operazioni da eseguire e l'elenco dei file sulle quali esse operano. Grunt � una applicazione modulare, supporta infatti l'utilizzo di svariati plugin per l'esecuzione di specifici task, come il \textit{linting} (controllo della sintassi) dei file di progetto, la \textit{minificazione} degli script CSS e l'esecuzione dei test unitari tramite l'utilizzo di apposite librerie.  
\item[Bower] si occupa della gestione delle dipendenze. Esegue il \textit{download} della libreria desiderata, se necessario, decomprime il file ottenuto e ne posiziona il contenuto in una specifica directory del progetto, in modo da renderlo fruibile all'interno dell'applicazione. Pu� accadere che il funzionamento della libreria scaricata, a sua volta, dipenda dalla presenza di altre librerie, dette \textit{dipendenze}. In questo caso, Bower eseguir� un controllo delle versioni di ogni libreria richiesta e avvertir� l'utente in caso di possibili incompatibilit�.\\
\end{description}

\begin{figure}[h]
\centering
\includegraphics[width=0.7\linewidth]{./img/yeoman.png}
\caption[Il logo di Yeoman]{Il logo di Yeoman}
\label{fig:yeoman}
\end{figure}

Per lo sviluppo di Mole.io � stato utilizzato un generatore di applicazioni AngularJS. Dopo aver lanciato il comando di generazione, Yeoman ha prodotto un progetto vuoto, ma completamente funzionante e predisposto per l'esecuzione di test unitari con l'uso di \textit{Karma} (un framework messo a disposizione dagli stessi creatori di AngularJS) e un server Node.js per servire staticamente i file contenenti l'applicazione. Grunt � stato anch'esso automaticamente configurato con task per l'esecuzione dei test, la minificazione, e l'avvio del server di supporto. Bower � stato utilizzato per ottenere le dipendenze di base quali \textit{jQuery.js}, \textit{Modernizr}, \textit{Less}.