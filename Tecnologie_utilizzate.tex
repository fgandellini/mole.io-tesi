In questo capitolo approfondiremo i dettagli delle tecnologie utilizzate per realizzare Mole.io. Illustreremo, per ognuna di esse, le motivazioni che ci hanno spinto alla scelta di particolari soluzioni software e le problematiche incontrate durante il loro utilizzo.

Mole.io � stato interamente sviluppato utilizzando il linguaggio JavaScript. L'utilizzo di questa tecnologia � abbastanza comune all'interno delle pagine web e si presta bene all'utilizzo \textit{client side}, meno comune � invece la sua applicazione nella parte \textit{server}. La piattaforma Node.js permette di utilizzare JavaScript per sviluppare la parte \textit{backend} delle applicazioni. Rendere uniforme il linguaggio utilizzato permette facilitare lo sviluppo e le fruibilit� del progetto da parte degli sviluppatori. Per poter lavorare al progetto � infatti richesta solo la conoscenza di JavaScript e non di altri linguaggi, questo � un ottimo requisito quando si pensa ad un team di sviluppo in espansione.

Quella del paragrafo precedente � solo una delle motivazioni che ci hanno spinto a scegliere Node.js come tecnologia di sviluppo, iniziamo quindi a vedere in dettaglio questa tecnologia.